%
% Documento: RESUMO (Inglês)
%

\begin{ABSTRACT}
	\begin{SingleSpace}
				
		\hspace{-1.5cm}The educational game \textit{Araguaia: The Saga of Osvaldão} for the mobile platform (Android), began to be developed in august of $2016$ by the team of the Laboratório de Games Educativos (LAGE) of the Universidade Federal do Sul e Sudeste do Pará (UNIFESSPA), with the purpose of transforming a regional historical narrative -- Araguaia's Guerrilla -- into an educational game idealized to be used as an auxiliary tool in the teaching-learning process of the History discipline for high school classes schools in Marabá-PA. The story of the game was built in the action of the guerrilla Osvaldão, because this had a very important role in the Guerrilla. The project of the game provides for two phases -- \textit{$1st$ phase: period of insertion of Osvaldão and his companions guerrillas in the region of ``Bico do Papagaio"\space $(1966-1971)$; and $2nd$ phase: a battle period of Guerrilla $(1972-1974)$}, each consisting of cutscenes and missions. This paper presents the design and implementation of a one cutscene in 3D, three cutscenes in 2D and four missions, two first related to the first phase, and two the second phase. The scenarios, objects and characters were modeled by other members of the LAGE team. The cutscene 3D was designed using the Blender software, for the three-dimensional modeling of the environment, and the Makehuman software, for the creation of humanoid characters. The cutscenes 2D were implemented using the Unity -- development engine that provides pioneering features for creating games and other interactive content. The four missions mentioned above were also implemented using Unity. Later, the game was evaluated by the team and presented to students of a school in Marabá-PA, obtaining encouraging results.
		
		\vspace*{0.5cm}\hspace{-1.5 cm}\textbf{Keywords}: Educational Game. 2D Plataform. Araguaia's Guerrilla. Osvaldão.
				
	\end{SingleSpace}
	
\end{ABSTRACT}
