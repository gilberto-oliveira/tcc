%
% Documento: RESUMO (Português)
%

\begin{RESUMO}
	\begin{SingleSpace}
		
	\hspace{-1.5cm}O jogo  educativo \textit{Araguaia: A Saga de Osvaldão} para a plataforma \textit{mobile (Android)}, começou a ser desenvolvido em agosto de $2016$ pela equipe do Laboratório de Games Educativos (LAGE) da Universidade Federal do Sul e Sudeste do Pará (UNIFESSPA), com o intuito de transformar uma narrativa histórica regional -- Guerrilha do Araguaia -- em um game educativo idealizado para ser utilizado como ferramenta auxiliar no processo de ensino-aprendizagem da disciplina de História para turmas do ensino médio das escolas de Marabá-PA. O enredo do jogo foi construído na atuação do guerrilheiro Osvaldão, pois este teve um papel muito importante na Guerrilha. O projeto do jogo prevê duas fases -- \textit{$1ª$ fase: período de inserção de Osvaldão e seus companheiros guerrilheiros na região do ``Bico do Papagaio"\space $(1966-1971)$; e $2ª$ fase: período de combate da Guerrilha $(1972-1974)$}. Cada uma delas é constituída por \textit{cutsecenes} e missões. Este trabalho apresenta o projeto e implementação, de uma \textit{cutscene} em \textit{3D}, três \textit{cutscenes} em \textit{2D} e de quatro missões, duas primeiras relacionadas a primeira fase, e duas a segunda fase. Os cenários, objetos e personagens foram modelados por outros membros da equipe do LAGE. A \textit{cutscene} \textit{3D} foi concebida usando o software \textit{Blender}, para a modelagem tridimensional do ambiente, e o software \textit{Makehuman}, para a criação dos personagens humanoides. As \textit{cutscenes 2D} foram implementadas utilizando a \textit{Unity} -- motor de desenvolvimento integrado que fornece funcionalidades pioneiras para a criação de jogos e outros conteúdos interativos. As quatro missões supracitadas também foram implementadas utilizando a \textit{Unity}. Posteriormente, o jogo foi avaliado pela equipe e apresentado para alunos de uma escola de Marabá-PA, obtendo resultados animadores.
	
	\vspace*{0.5cm}\hspace{-1.5 cm}\textbf{Palavras-chave}: Jogo educativo. Plataforma \textit{2D}. Guerrilha do Araguaia. Osvaldão.
	
	\end{SingleSpace}
\end{RESUMO}