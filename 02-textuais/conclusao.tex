% CONCLUSÃO--------------------------------------------------------------------

\chapter{CONSIDERAÇÕES FINAIS}
\label{chap:conclusao}

Este trabalho descreveu o projeto e implementação de partes componentes de um jogo de plataforma \textit{2D} sobre um acontecimento recente da história brasileira, a Guerrilha do Araguaia. Foi apresentado o processo de desenvolvimento de quatro \textit{cutscenes}, uma em \textit{3D} e três em \textit{2D}, e quatro missões. Duas missões relacionadas a primeira fase do jogo, e duas a segunda fase. Foi formada uma equipe interdisciplinar com profissionais da área de história e da área de produção de games. Inicialmente a equipe passou por um processo de ambientação sobre o assunto a ser relatado na forma de um game, e trabalhou de forma colaborativa na escolha do estilo do jogo e na definição do seu enredo e roteiro. Os resultados, até aqui alcançados com a finalização das duas fases são animadores e motivadores. É um jogo que une diversão e que ao mesmo tempo instrui não só sobre os acontecimentos históricos, mas também sobre a flora e fauna do palco dos acontecimentos.

A distribuição do jogo foi feita em dois aplicativos. O primeiro aplicativo, que se refere a primeira fase do jogo, onde o período de ambientação na região é retratado. E o segundo aplicativo, que mostra o relacionamento com a população e treinamento dos guerrilheiros; e o conflito armado (tabela \ref{tab:enrrotmis}).

Inicialmente, o primeiro aplicativo foi construído na plataforma \textit{desktop} e testado com boa aceitação nas turmas do primeiro ano do ensino médio de um colégio particular. A equipe tentou levar esta versão, também para escolas públicas, mas encontrou dificuldades devido a baixa qualidade dos computadores, tanto de hardware como de software, onde usam um sistema operacional \textit{Linux} antiquado e principalmente a falta de manutenção torna muito difícil a utilização do jogo por parte de alunos e professores.

Diante disso, a equipe constatou que a grande maioria dos alunos das escolas públicas possuem aparelhos celulares com capacidade suficiente para rodar jogos no estilo plataforma \textit{2D}, e decidiu que o jogo seria distribuído para \textit{mobile} \textit{(Android)}, por ser a mais popular entre os alunos das escolas públicas. Assim, os dois aplicativos estão construídos para dispositivos móveis, disponíveis para \textit{download} em \url{https://lage.unifesspa.edu.br/baixar-games-main.html}. Acredita-se que, o que foi construído neste trabalho reúne material suficiente para ser apresentado como um game educativo lúdico sobre um fato histórico tão pouco conhecido por professores e alunos.

\section{Trabalhos Futuros}

Como trabalho futuro, sugere-se a realização da avaliação com alunos (descrita na seção \ref{sec:desenvolvimento}) em algumas escolas do município de Marabá-PA, para evidenciar quantitativamente a receptividade do jogo por parte dos envolvidos, tal como a experiência destes, por meio da aplicação de um questionário extraído de um modelo para avaliação de jogos educacionais, descrito em \citeonline{bib:savi2011}.

\section{Artigos Publicados}

\begin{itemize}
	\item OLIVEIRA, Gilberto P.; RESPLANDES, Denison S.; SILVA, Caique; SOUZA, Adriano; ARAÚJO, Tiago de S.; LUIZ, Janailson M.; RIBEIRO, Manoel F. Guerrilha do Araguaia: A atuação do guerrilheiro Osvaldão em um Jogo de Plataforma que conta um episódio histórico Pouco Conhecido. Publicado no $XVI$ SBGames -- Simpósio Brasileiro de Jogos e Entretenimento Digital, realizado em novembro de $2017$, em Curitiba-PR (Apêndice \ref*{chap:apendiceA} e Anexo \ref*{chap:anexoA}).
	
	% \item OLIVEIRA, Gilberto P.; RESPLANDES, Denison S.; RIBEIRO, Manoel F.; LUIZ, Janailson M.; SOBRAL, Marcos; NOGUEIRA, Felipe. Araguaia -- A Saga de Osvaldão: um jogo educativo sobre a Guerrilha do Araguaia para a plataforma \textit{Android}. Publicado no SBIE $2018$, $XXIX$ Simpósio Brasileiro de Informática na Educação (SBIE -- Trilha $2$: Jogos, simulação, gamificação, robótica, realidade virtual e mundos virtuais para promoção da aprendizagem), com aceite previsto para meados de agosto de $2018$.
\end{itemize}

